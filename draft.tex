\section{Circular Economy}

\begin{itemize}
\item \textbf{Linear economy} : Take $\rightarrow$ Make  $\rightarrow$ Distribuite $\rightarrow$ Consume  $\rightarrow$ \textbf{ Dispose}
\item \textbf{Circular Economy} :  Take  $\rightarrow$ Make  $\rightarrow$ Distribuite $\rightarrow$Consume  $\rightarrow$  \textbf{Return}
\end{itemize}



\begin{itemize}
	\item Undestand $\rightarrow$ How to source what it needs?
	\item Manage  $\rightarrow$ How to create Value?
	\item Develop  $\rightarrow$ What kind of buisiness model?
\end{itemize}

Il problema principale di un economia lineare è che si crea una scarsità di risorse e una quanitità di rifiuti tossici per l'ambiente.

\textbf{Sustainable development}
It is seen as the process of satisfying the current needs of the population without compromising the
capacity to do so of future generations.

\textbf{Sustainability}
It is often understood as the protection of non-renewable natural resources, biodiversity and avoidance
of climatic changes.
With the main focus being on environmental issues this type of sustainability is also framed as ecological
sustainability. There is also social sustainability defined as sustainability that “refers to actively supporting
the preservation and creation of skills as well as the capabilities of future generations, promoting health
and supporting equal and democratic treatments that allow for good quality of life both inside and
outside of the company context.

Sustainability can be seen as part of sustainable development and the latter term describes
the transition process towards a sustainable world.
The circular economy can be understood as a concept contributing to this transition process.

The sustainability revolution: the three pillars
It is based on the often called three E’s:
\begin{itemize}
\item ecology/ environment,
\item economy/ employment,
\item equity/ equality
\end{itemize}

The sustainability revolution requires a transition implying changes in technologies,
infrastructure, lifecycles and institutions as well as in structures of consumption and production
that are radical, long-term and far-reaching.

This means that a system innovation for the sustainability goal is required.
A system innovation can be described as a combination of innovations on various levels to
provide service in a new way.
This entails a socio-technical change with new ways of practice and consumption. The term
socio-technical in this context means that this change does not only affect a technological
change but in addition it requires a modification of social patterns. These fundamental shifts
usually take several decades and require an interplay between a variety of actors such as
policy makers, knowledge-generating institutions, companies and customers.

\subsection{Does sustainability mean to limit the growth?}
The question should not be if economic growth can be combined with environmental concerns but
how growth can be combined with preservation of the environment.
Resource productivity and eco-efficiency play an important role in the context of environmental
concerns and lead to the concept of decoupling.
This concept has the objective of reducing resource depletion and environmental impact while
ensuring economic growth (United Nations Environment Program 2011).
Decoupling can be understood as an important factor to ensure long-term economic growth
under the condition of sustainable development (using less material, energy, water as well as
land resource and reusing material)The UNEP International Resource Panel defines two aspects of the decoupling:
The resource decoupling means reducing the rate of use of resources per unit of
economic activity, and the impact decoupling means maintaining economic output while
reducing the negative environmental impact of the underlying economic activities.

\subsection{Circular Economy }

\newline an economy that provides:
multiple VALUE creation
mechanisms which are decoupled
from the consumption of FINITE
RESOURCES
\begin{itemize}
\item Design out waste
\item Build resilience
through diversity
renewable sources
\item  Think systems cascades
\item The CE thinking requires the application of 3 PRINCIPLES that together
lead to an economy that is prosperous while being natural capital
restorative and regenerative
\begin{itemize}
	\item preserve and enhance natural capital
	\item  optimase resourse yield (Tehnical and Biological cycles)
	\item foster system effectiveness (no - env esternalities)
\end{itemize}
\end{itemize}

\subsection{Transition Tu circular economy}
Four important blocks:
\begin{itemize}
\item materials and product design,
\item new business models,
\item  global reverse networks, and
\item enabling conditions :Education ,Financing, Collaborative platform , New economic framework for pricing externalities
\end{itemize}

\subsection{Circular Business Model }
Business Model: the rationale of how an organization creates,
delivers and captures value.

Purpose “profit/value/ well-being”
describes the design/architecture/framework
of the mechanisms employed of value.

\textbf{Circular Business Model}
Strategic systemic thinking to gain Circular advantages
Innovate :Productivity of resources ,
Value within the whole life cycle of the product.

\subsubsection{Example of Circular Business Model}
\begin{enumarate }
\item \textbf{Circular supply chain }
\begin{itemize}
\item Circular process : regain Value from components/waste/products at
the end of their cycle:closed: product from inside the
company , open: product from other companies
\item Objective : Gain access to renewable and recycled
Inputs
\item advantages
More independence from external
resources (decouple production from
shocks, resource unavailability,
volatility of price.
\end{itemize}
\item  Extension of lifecycel
\item sharing 
\item Product as a Service
\end{enumarate }

\subsection{Policies for a circular economy}


